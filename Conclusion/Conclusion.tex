%!TEX root = ../thesis.tex
%*******************************************************************************
%*********************************** First Chapter *****************************
%*******************************************************************************

\chapter*{Conclusions}  %Title of the First Chapter
\addcontentsline{toc}{chapter}{Conclusions}

\ifpdf
    \graphicspath{{Conclusion/Figs/}{Conclusion/}}
\else
    \graphicspath{{Conclusion/Figs/}{Conclusion/}}
\fi

The natural process of societal evolution has brought many changes in the political, social, economic and environmental contexts. These changes were mainly a result of technological progress that spurred an improvement in the use of the environment and consequently led to its degradation. Considered one of major the causes of climate change, deforestation releases billions of tonnes of carbon dioxide and other greenhouse gases into the atmosphere and causes the biodiversity loss in the tropical regions and is damaging the environmental system in the planet.

The purpose of this thesis has been to examine the economic determinants of deforestation in Brazil and the effectiveness of environmental policies taking place in the country using innovative and interdisciplinaries techniques. We analysed first the institutional environmental framework (IEF) in the economic delimitation of Brazilian Amazon (Legal Amazon) controlling for market expansion which was characterised the main driver of deforestation at the time of study. The results from the first chapter suggest that the Institutional Environmental Framework conditioned on policies and prices curbed deforestation within municipalities. Since deforestation has a spatial dimension we expand the study to include spatial analysis. We observe that the Institutional Environmental Framework when established in a municipality tend to reduce deforestation in neighbouring municipalities. An anecdotal counter-factual simulation indicated that the existence of institutional environmental framework avoided forest clearing that would have occurred had the institutional framework not been implemented. 

Following the results from the first chapter, we focused our analysis on the deforestation trends observed in the Ecological Tension Zone of Maranhao, which provided us with unique natural experiment in that there were spatially heterogeneous environmental policies to combat deforestation. To understand the deforestation trends in that region we used non-linear modelling for the task since it is recognised that most ecological and climatic data represent complex relationships and thus non-linear models, such as Generalized Additive Models (GAMs), may be particularly suited to capture confounding effects in trends. Our findings suggest that deforestation is related to year and several climatic covariates, but also revealed that there are substantially differences in trends between seasons and regions. For the region under a surveillance system most of the deforestation happened during the rainy season and, for the region not under the monitoring policy, there was a well-defined deforestation trend for both seasons. 

Finally, in chapter 3 we combine the findings from the previous chapters to elaborate on the motivation. Deforestation rates have declined in Brazil over the past two decades and it is believed that environmental policies conditioned on the institutional framework have played a crucial role. Moreover, the satellite monitoring program has enabled authorities to identify and react to deforestation in a much more timely manner than local field detection. Given that the trends of deforestation in two regions, under different environmental policies, of an ecological tension zone (ecotone forest) in Maranhão showed diverged results. We assumed that cloud coverage, by delaying detection until skies are clear, has acted as an important impediment to the policy's success. Focusing on the ecological tension zone of Maranhão that is separated into two parts by an artificial line- one that was covered by environmental deforestation policy and the other that is not subject to this -  we use satellite data within a survival analysis framework to estimate how the probability of transition between intact forest to disturbed forest, given risk factors and conditional on the time elapsed until the occurrence of the transition, is affected by cloud coverage. Our findings suggest that the presence of clouds has increased deforestation in the region covered by the satellite detection program, and thus was likely an active barrier to legal compliance. 

Overall, our results has policy implications for environmental policies in Brazil. We've seen that the institutional environmental framework is important for the protection of the tropical forests when combined with environmental enforcement. The actual institutional framework needs to be tightened up in terms of implementation. Most important, it is pertinent that the established efforts proceed in the Brazilian Amazon, in spite of any political changes in Brazil and, strengthening the institutional framework must be detached from any transient actions. We then, observed the past trend of deforestation in two areas of great importance for the Brazilian biome, Amazon and Cerrado and, the results indicate that the environmental policies in the Legal Amazon should be expanded to ecotonic/transition forests along the Amazon Forest because these forests represent the first indication of anthropic intervention. Also, we believe that the deforestation monitoring system should be improved by the use of satellites that are not constrained by climatic events such as cloud cover. Finally, it is important to acknowledge that significant deforestation is happening in areas of transition between Amazon and Cerrado and, the implementation or expansion of satellite monitoring program must be applied to further biomes like Cerrado, the second most degraded biome in the country.

\textbf{LIMITATIONS} Although our results presented in this thesis are in line with previous studies and our empirical evidence has been corroborated by robustness checks and model validations, our analysis still suffers from a number of weaknesses. 


%Chapter 1
Our main results in the first chapter might suffer from the issue of omitted variable bias. We tried to include all possible variables that could potentially affect deforestation but data limitations prevented us to capture for all possible determinants. To deal with this issue we looked for proxies that could minimise this problem. Another possible issue is the potential endogeneity of many of the explanatory variables, and hence their interpretation in terms of causality.  Since it would be difficult to find plausible instruments for many, if not for all, of our independent variables, we instead tried to control for municipality fixed effects, allowing us to purge all time invariant unobservables from the specifications and, we lagged all control variables by one period, so that under assumption that, after controlling for fixed effects all confounding shocks are only contemporaneous in nature, we are left with solely exogenous variation.  


%chapter 2
In the second and third chapters, we a have a number of limitations that must be taken into account. First, the model implicitly assumes that the predicted range or potential space is fully occupied by forest, which in reality might not be true. Secondly, the spatial distribution of the vegetation indices may exhibit dynamic behaviour over time, so that a potential area may or may not be sparsely vegetated for a certain period (e.g., during sampling) due to progressive succession of forest. Or a temporary absence could be due to natural causes, such as, an attack of pests or diseases or inter-species competition. Thirdly, the study was based on coarse image resolution which could neglect local changes in the sample area. Finally, our results may not  be generalised to other areas, such as dense tropical forest and open fields. The same issues for chapter two apply to chapter three since we use the same dataset. In addition, for the third chapter, we only considered for the analysis covariates before or from 2000, which thus excluded roads, protected areas, indigenous land, markets, municipalities centre, and, mining/mineral resources created or discovered after 2000. Finally, we acknowledge that the models were derived from NDVI values and that one could alternatively have used EVI values, which in some instances might be more suited for ecotone forests.


\textbf{FUTURE RESEARCH} This thesis represent a starting point for a research agenda which can be extended in the future. First, departing from the analysis of deforestation in Brazil, it would be interesting to look at other tropical countries investigating the role of environmental policies taking into account the different settings of the institutional framework. There are data available on the proxy of deforestation for many tropical countries and different policies approach have been undertaken for different countries, such as Bolivia, Colombia and, Venezuela. For this reason, it would be interesting to conjecture the differences existing with different institutional apparatus. Secondly, the estimation of the deforestation trends used coarse resolution to the analysis. However, a more in-depth analysis using fine resolution might be needed given the increasing action of selective logging that cannot be captured observing at a coarse resolution. Finally, our analysis of chapter 3 has shown the relevance of climatic events as an impediment of satellite monitoring program effectiveness. We could corroborate the findings by applying a fine resolution satellite which is not affected by cloud cover to reassure our findings.